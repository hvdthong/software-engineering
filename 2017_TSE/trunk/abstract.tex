Developers often spend much effort and resources to debug a program. To help the developers debug, numerous information retrieval (IR)-based and spectrum-based bug localization techniques have been devised. IR-based techniques process textual information in bug reports, while spectrum-based techniques process program spectra (i.e., a record of which program elements are executed for each test case). While both techniques ultimately generate a ranked list of program elements that likely contain a bug, they only consider one source of information---either bug reports or program spectra---which is not optimal. In light of this deficiency, this paper presents a new approach dubbed \underline{Net}work-clustered \underline{M}ulti-modal Bug \underline{L}ocalization (NetML), which utilizes multi-modal information from both bug reports and program spectra to localize bugs. NetML facilitates an effective bug localization by carrying out a joint optimization of bug localization error and clustering of both bug reports and program elements (i.e., methods). The clustering is achieved through the incorporation of \emph{network Lasso} regularization, which incentivizes the latent parameters of similar bug reports and similar program elements to be close together.  To estimate the latent parameters of both bug reports and methods, NetML features an adaptive learning procedure based on Newton method that updates the parameters on a per-feature basis. Extensive experiments on 157 real bugs from four software systems have been conducted to evaluate NetML against various state-of-the-art localization methods. The results show that NetML surpasses the best-performing baseline by 48.39\%, 15.49\%, 8.7\%, and 13.92\%, in terms of the number of bugs successfully localized when a developer inspects the top 1, 5, and 10 methods and Mean Average Precision (MAP), respectively.

%In addition to using bug report text and program spectra features (which are akin to the IR-based and spectra-based techniques respectively), NetML also incorporates a hybrid \emph{method suspiciousness} feature that stems from processing both bug report and program spectra information.

%N-AML \textit{adaptively} creates a bug-specific model to map a particular bug to its possible location, and introduces a {\em suspicious words} that are highly associated to a bug. N-AML is an extension of \underline{A}daptive \underline{M}ulti-Modal bug \underline{L}ocalization (AML) which was presented in Foundations of Software Engineering Conference. More specifically, N-AML involves two extensions compared to AML: redefine the function of suspiciousness score between bugs and program elements, and make use of Network Lasso regularization to exploit the similarities amongst different bugs and program elements to localize the bug among list of program elements.  

%We evaluate AML* on 157 real bugs from four software systems, and compare it with AML and state-of-the-art bug localization approaches.
%Experiments show that AML* outperforms the best baselines by at least 119.05\%, 51.85\%, 38.89\%, and 46.74\% in terms of number of bugs successfully localized when a developer inspects 1, 5, and 10 program elements (i.e., top 1, top 5, and top 10), and Mean Average Precision (MAP) respectively. Moreover, the results also show that AML* also outperforms AML by 48.39\%, 15.49\%, 8.7\%, and 13.92\% when the developer investigates top 1, top 5, top 10 methods and MAP respectively. 

%We evaluate N-AML on 157 real bugs from four software systems, and compare it with original multi-modal techniques (i.e. AML). Experiments show that N-AML outperforms AML by at least 48.39\%, 15.49\%, 8.7\%, and 13.92\% in terms of number of bugs successfully localized when a developer inspects 1, 5, and 10 program elements (i.e., top 1, top 5, and top 10), and Mean Average Precision (MAP) respectively. Moreover, we also compare N-AML with the state-of-the-art bug localization approaches. The results demonstrate that N-AML surpasses the best baseline (i.e. PROMESIR) by  119.05\%, 51.85\%, 38.89\%, and 46.74\% in top 1, top 5, top 10 programs, and MAP respectively.

%successfully localize 92 out of the 157 bug reports, which is 27.78\% more than the best baseline, when developers only inspect the top-10 recommended program elements. In terms of Mean Average Precision (MAP), our approach outperforms the best baseline by 28.80\%.

%considering only bugs that can be localized by analyzing the first 10 recommended program elements in a ranked list, our approach successfully localizes faulty methods  in 92 out of 157 bug reports.our approach can achieve a mean average precision (MAP) score of {\color{red}{0.237}} which is substantially higher than the results of the baseline approaches.

%our proposed approach can achieve a success-rate@1 of {\color{red}17.83\%} which is {\color{red}33.33\%} higher than the best baseline.

%These techniques can be grouped to two main families: those that use information retrieval and processes textual description in bug reports to find buggy program elements that are textually similar to the bug reports, and those that analyze failing and correct test case executions and compute suspiciousness scores of program elements that are executed more in the failing rather than successful executions.

